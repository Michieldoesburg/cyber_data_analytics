\documentclass[]{article}
\usepackage{graphicx}

%opening
\title{Assignment 1 CDA}
\author{Tim van Rossum, 4246306\\
	Michiel Doesburg,}

\begin{document}

\maketitle

\section{A visualization of the data}
For the visualization part of this assignment, we first started out by making bar plots of different kinds, but eventually settled  for the scatterplot as seen in figure 1. This scatterplot uses the amount of Eurocent spent per transaction, to allow for better comparisons to be made, as there were five different currencies in the dataset. As can be seen from the scatterplot, there are basically no fraudulent transactions where more than 800 Euro was spent, while there are benign transactions where more than 800 Euro was spent.
\begin{figure}[h!]
	\centering
	\includegraphics[scale = 0.25]{Visualizations/fraud_vs_nonfraud_better}
	\caption{A scatterplot of the amount of money spent in the sampled transactions. A red dot indicates a fraudulent transaction (these are only at the very left of the plot due to very little fraudulent transactions existing), while a blue dot indicates a benign transaction.}
\end{figure}
\clearpage
\section{Applying SMOTE to the data}
Because we used Python for this assignment with \texttt{scikit-learn}, we used SMOTE as implemented in the package \texttt{imblearn}. The \texttt{fraud\_detection.py} script already preprocessed the data in such a way that applying SMOTE to it was very easy, as the implementation only needed the data and the class labels, and both were already generated by the script. The classifiers that we used were the random forest classifier, the 5-NN classifier, and the logistic classifier. Points on the ROC curve were generated by varying the size of the training set and varying the parameters of the classifiers.
\end{document}
